\chapter{Contexto}
\label{context}

\section{Sobre a Engrena}

O Movimento Empresa Júnior (MEJ) faz parte de uma iniciativa dos estudantes de estabelecer ambientes onde serão praticados os conceitos aprendidos em sala de aula. Dessa forma, o movimento tem como objetivo a capacitação profissional dos seus membros, além da iniciação no mundo dos negócios, envolvendo todas as questões que estariam presentes em uma situação real de negócio. Toda essa ideologia é posta em prática através das empresas juniores.

A Engrena é uma empresa que adota esse movimento, ou seja, é uma empresa júnior do curso de Engenharia Automotiva, oferecido na Universidade de Brasília (UnB), na Faculdade do Gama (FGA). A mesma é dividida em cinco partes, presidência, gestão de pessoas, gestão de qualidade, projeto e administração financeira. Almejando o objetivo de “oferecer consultorias para oficinas mecânicas e empresas em geral, além de auxiliar no desenvolvimento de diferentes projetos dentro da área”, como é encontrado em sua página oficial em uma rede social.

Além disso, a Engrena valoriza o aprendizado profissional dentro de seu ambiente e para isso, adota práticas para maximizar essa variável. Uma das práticas, por exemplo, é tentar mudar os integrantes de áreas internas a cada seis meses, com o objetivo de rotacionar o conhecimento da empresa entre os membros, pedindo que os mesmos façam um relatório sobre o que foi aprendido durante esse tempo. 

\section{Contexto do Trabalho}

A empresa tem tido dificuldades em manter uma rastreabilidade de seu pessoal e suas respectivas funções. Há algum tempo, todos os diretores da Engrena estavam em estágios avançados de seus cursos ou em formação. Isso impossibilitou que a atenção devida pudesse ser dada a atividades gerais. Por exemplo, muitos membros que mudavam de área não faziam os relatórios. Indo mais além, muitas atividades dentro da empresa não estavam sendo cumpridas como estipulado.

Dessa forma, pode-se concluir as seguintes observações:

\begin{itemize}
\item Objetivo da empresa: Conseguir manter a rastreabilidade do pessoal interno, com seus respectivos cargos e áreas dentro da empresa. Além disso, é esperado uma forma de avaliar o desempenho individual e de uma área da empresa através da razão entre o número de atividades realizadas e o número total de atividades que eram esperadas de tal indivíduo/área, através de um \textit{software} ainda não especificado que seja capaz de fornecer funcionalidades que supram as necessidades citadas nos itens anteriores.
\end{itemize}
