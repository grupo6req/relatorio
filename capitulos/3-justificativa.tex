\chapter{Justificativa da Abordagem}
\label{justify}

\section{Principais abordagens utilizadas no mercado}

Em desenvolvimento de software, há as abordagens dos modelos preditivos e adaptativos. Cada modelo tem suas vantagens e desvantagens, se encaixando melhor em contextos determinados. Para a construção desse projeto, foram analisados os processos \textit{Rational Unified Process} (RUP) e \textit{Scaled Agile Framework} (SAFe) para verificação e otimização do mais adaptado ao contexto encontrado.

\section{Sobre o RUP - \textit{Rational Unified Process}}

O \textit{Rational Unified Process} é um processo de engenharia de \textit{software} que provê disciplinas e atribui tarefas e responsabilidades para uma organização desenvolvedora, que procura um modelo visando qualidade e desenvolvimento iterativo.

\subsection{Estrutura do RUP}

A sua estrutura pode ser observada pela figura 1, onde:

\begin{itemize}
\item O eixo horizontal representa tempo e aspectos do ciclo de vida do projeto de \textit{software}
\item O eixo vertical representa as disciplinas implementadas no processo de \textit{software} e sua magnitude em cada etapa do ciclo de vida.
\end{itemize}

O processo unificado se divide nas seguintes fases:

\begin{itemize}
\item Iniciação: “Envolve tanto a atividade de comunicação com o clienta quanto a de planejamento. Colaborando com os interessados, identifica-se de negócio para o software; propõe uma arquitetura rudimentar e se desenvolve um planejamento para a natureza iterativa e incremental do projeto decorrente...”
\item “Envolve atividades de comunicação e modelagem do modelo de processo genérico. A elaboração refina e expande os casos práticos preliminares desenvolvidos como parte da fase e iniciação, e amplia a representação da arquitetura...”
\item Construção: “Tendo como entrada o modelo de arquitetura, a fase de construção desenvolve ou adquire componentes de software; esses componentes farão com que cada caso de uso se torne operacional para o usuário final. Então implementa-se, no código-fonte todos os recursos e funções necessárias  e exigidas para o incremento do software. À medida que os componentes estão sendo implementados, desenvolve-se e executam-se testes unitários para cada um deles. Além disso, realizam-se atividades de integração…”
\item Transição: “Abarca os últimos estágios da atividade de construção genérica e a primeira parte da atividade da atividade de emprego genérico: entrega e feedback. Entrega-se o software ao usuários finais para testes beta e coleta de seu feedback para respectivas mudanças e bugs encontrados. Alem disso a equipe de software elabora material de apoio (manuais, guias e etc) que são necessários para o lançamento da versão. Na conclusão da fase de transição, o incremento torna-se uma versão de software utilizável.”
\end{itemize}

\section{Sobre o SAFe - \textit{Scaled Agile Framework}}

A metodologia ágil existe desde a década de 80, porém foi mal interpretada durante muito tempo. Foi confundido seu foco nas pessoas com ausência de documentação. Por essa razão, muitos desenvolvedores achavam que a metodologia era descuidada, sem padrão ou documentação. Essa informação não é verídica e há exemplos de metodologia ágil bem sucedida na indústria, como por exemplo na Toyota, que adota uma linha de produção nesse modelo. 

O SAFe se baseia em princípios Lean e Ágeis para sintetizar o conhecimento obtido por gerações de deployments de softwares em um único framework, fornecendo um conjunto integrado de atividades que têm demonstrado melhorar aspectos no desenvolvimento de software, como por exemplo produtividade do time e qualidade da solução.

Tendo como base a figura acima, é possível perceber que as atividades associadas à requisitos no SAFe estão divididas em três visões:
\begin{description}
\item [Portfólio] É o nível mais alto do SAFe, onde estão alinhadas as estratégias de negócios da empresa e intenções de investimento da empresa. Nesse nível estão alguns pontos chaves, como: épicos, \textit{portfolio backlog} e métricas.
\item [Programa] Nível responsável por entregar frequentemente incrementos no programa, tipicamente entre 8-12 semanas em múltiplas iterações. Está associado, portanto, ao \textit{Agile Release Train}, que pode ser visto na figura.
\item [Time] Na visão de time, são fornecidas abstrações organizacionais para fornecer papéis e atividades que nortearão o desenvolvimento de \textit{software} em grandes empresas. 
\end{description}

\section{Abordagem escolhida}

Várias razões motivaram a optar pela a abordagem iterativa incremental baseada no \textit{Rational Unified Process}, descartando abordagens ágeis ou híbridas, sendo:
\begin{itemize}
\item Em relação a abordagem ágil, a equipe possui pouca disponibilidade de tempo para realizar \textit{stand-ups} diários, retrospectiva, escrever histórias de usuários junto ao cliente, revisão com os \textit{stakeholders} e pareamento, principais práticas ágeis. Dessa forma, evita-se a utilização da metodologia ágil ou híbrida, pois haveria o prejuízo na adoção das mesmas.
\item Os integrantes da equipe se sentem mais confiantes quando estão sob uma metodologia de desenvolvimento com uma ampla utilização no mercado e documentação sobre suas práticas.
\item Como o \textit{Rational Unified Process} consiste em um modelo bem detalhado, facilitará um processo mais sutil da gerência de requisitos, porque nesse modelo a presença de documentos é fundamental.
\item A aborgadem híbrida foi evitada por ser concordado entre a equipe que é necessária experiência razoável para implantar os mecanismos das duas abordagens em um único processo de forma a conversarem harmonicamente. Mudanças simples como trocar histórias por casos de uso ou alguma outra troca de atividade não resultaria na integração e experiência desejadas para o desenvolvimento.
\item É predominante a abordagem iterativa incremental no mercado de trabalho de Brasília. Desse ponto de vista, é interessante utilizar a mesma para adquirir experiência.
\end{itemize}








