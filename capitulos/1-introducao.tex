\chapter{Introdução}
\label{introduction}

O Desenvolvimento de Software é dado através de passos definidos pela equipe que irá desenvolvê-lo levando-se em consideração diversos aspectos que podem influenciar o sucesso do projeto. Esse conjunto de passos é conhecido como processo de desenvolvimento. Um processo pode ser definido como um conjunto de atividades, ações e tarefas realizadas na criação de algum produto de trabalho.~\cite{pressman}

O presente contexto tem como foco a Engenharia de Requisitos. Os requisitos para um sistema são as descrições dos serviços prestados pelo sistema e suas restrições operacionais. [...] O processo de descobrir, analisar, documentar e verificar estes serviços e restrições é chamado de Engenharia de Requisitos (ER).

Este documento tem por objetivo apresentar os seguintes aspectos relacionados à ER que foram analisados e definidos pela equipe:
\begin{itemize}
\item Contexto do projeto: informações acerca da empresa alvo e seus desafios;
\item Justificativa da abordagem: razões para a escolha da abordagem definida pelo grupo;
\item Processo escolhido: atividades escolhidas para nortear o desenvolvimento do projeto e respectivos artefatos;
\item Elicitação de requisitos: técnicas escolhidas para facilitar o levantamento de requisitos;
\item Tópicos de gerência de requisitos: solução de rastreabilidade e modelo de maturidade adotados para dar suporte ao processo;
\item Planejamento do projeto: cronograma das atividades;
\item Ferramenta de gestão de requisitos: software escolhido para cadastramento e análise dos requisitos;
\end{itemize}

No encerramento do documento, há uma seção para as considerações finais da equipe sobre pontos específicos e uma para as referências bibliográficas utilizadas na elaboração deste texto. 
