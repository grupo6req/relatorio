\chapter{Considerações Finais}
\label{final-consideration}

Este projeto contribui para o melhor compreendimento da ER ao possibilitar colocar em prática o que é lecionado em de sala de aula. Dessa forma, os ensinamentos ganham experiência prática.

Houve uma dificuldade no começo do projeto, em que a comunicação com o cliente não estava conseguindo ser estabelecida. Algum tempo foi perdido nessa problemática, porém o progresso foi retomado ao ter sido designado para a equipe o representante da empresa.

Por fim, pode-se apenas reafirmar a fundamental importância da Engenharia de Requisitos no desenvolvimento de \textit{software}. A interação com o cliente possibilita uma simulação real de uma situação de trabalho com diversas variáveis e responsabilidades. É correto dizer que além do crescimento teórico, então, foi alcançado um crescimento pessoal graças ao aprendizado de bons comportamentos profissionais.
